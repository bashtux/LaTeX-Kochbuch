\begin{recipe}{Zander an Rahmkraut}{nach Dieter Müller}
  \inglist[Für den Fisch]
  \ingredient{1 Zander\index{Fisch>Zander}}
  \ingredient{1 EL Mehl}
  \ingredient{2 El Olivenöl}
  \ingredient{1 EL Butter}
  \ingredient{1 Thymianzweig}
  \ingredient{1 Rosmarinzweig}
  \ingredient{200 ml Fischsauce}

  \inglist[Für das Kraut]
  \ingredient{400 g Sauerkraut}
  \ingredient{1 Zwiebel}
  \ingredient{100 ml Weißwein}
  \ingredient{200 ml Geflügelfond}
  \ingredient{1 Apfel}
  \ingredient{1 Kartoffel}
  \ingredient{2 EL Sahne}
  \ingredient{200 g Gemüse}
  
  \inglist[Für den Gewürzbeutel]
  \ingredient{5 Wachholderbeeren}
  \ingredient{2 Nelken}
  \ingredient{5 Pimentkörner}
  \ingredient{10 Pfefferkörner}
  \ingredient{1 Lorbeerblatt}
  \ingredient{1 Thymienzweig}
  
  \steps
  den Zander schuppen, waschen und filetieren. Restliche Gräten herausziehen, die Filets
  protionieren und kalt stellen. Von den Resten einen Fischfond bereiten.

  Das Sauerkraut mit den gewürfelten Zwiebeln in etwas Butter andünsten. Anschließend
  Wein, Fond und den Gewürzbeutel zugeben und etwa 20 Minuten gar ziehen lassen. Apfel und
  Kartoffel schälen, würfeln und fein pürieren. Damit das Kraut binden und nochmals 2
  Minuten köcheln lassen, dann mit Salz und Pfeffer abschmecken.

  Die Zanderfilets mir Salz und Pfeffer würzen, leicht mehlieren und in Olivenöl braten.
  Im letzten Moment mit Butter, Thymian und Rosmarin aromatisieren.

  Die gebratenen Gemüsewürfel zum Kraut geben und eventuell noch geschlagene Sahne
  zugeben. Die Fisch-Weißweinsauce (siehe Seite \pageref{Fischsauce}) mit Senf aufschäumen.
\end{recipe}
