\begin{recipe}{Lahmacun}{Von Lutz Moppert}
  \label{Lahmacun}
  \inglist[Für den Teig:]
  \ingredient{200 g Mehl}
  \ingredient{130 ml Wasser}
  \ingredient{10 g Hefe}
  \ingredient{\halb TL Salz}

  \inglist[Für die Sauce:]
  \ingredient{500 g Rinderhack}
  \ingredient{1 Zwiebel}
  \ingredient{1 Zehe Knoblauch}
  \ingredient{Petersilie oder Basilikum}
  \ingredient{1 rote Paprika}
  \ingredient{1 Pepperoni}
  \ingredient{1 EL Tomatenmark}
  \ingredient{1 TL Butter}
  \ingredient{2 TL Paprikapulver}
  \ingredient{2 TL Salz}
  \ingredient{1 TL Pfeffer}

  \steps

  Die Hefe im Wasser auflösen und die Hälfte des Mehls unterrühren. Das
  restliche Mehl darüber geben (noch nicht verrühren) und mit dem Salz mischen.
  Nach ca.  15 Minuten mit dem Handmixer mindestenz 10 Min. durchkneten. Den
  Teig in Portionen Teilen (ca. 100 g pro Pizza), zu Kugeln formen und mit
  Olivenöl einreiben.

  Den Teig mindestens eine halbe Stunde gehen lassen, dann noch einmal mit der
  Hand durchkneten bis das Olivenöl homogen verteilt ist, anschließend
  ausrollen und wieder 10 Minuten gehen lassen.

  Zwiebeln, Koblauch, Pepperoni und Paprika in winzige Würfel schneiden oder
  mit der Küchenmaschine zerkleinern. Petersilie hacken, mit den restlichen
  Zutaten der Sauce mischen und auf dem dem Teig verteilen. Die Pizza im
  vorgeheizten Backofen auf voller Stufe auf dem Boden liegend oder auf einem
  Pizzastein fertig backen.

\end{recipe}
