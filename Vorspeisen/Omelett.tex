\begin{recipe}{Omelett}{nach Christian Rach}
  \inglist
  \ingredient{3 Eier\index{Eierspeise}}
  \ingredient{40  ml Sahne}
  \ingredient{15 g Butter}
  \ingredient{2 EL Füllung}

  \steps
  Eier und Sahne in einer Schüssel mit der Gabel leicht verquirlen und mit Salz und
  Pfeffer würzen. Die Füllung (hierfür eignen sich z.B. gewürfelte Tomaten mit Basilikum
  oder gebratene Pilze mit Speck) vorbereiten und gegebenenfalls warm stellen.

  Butter in einer Pfanne zerlassen und die Eimischung hineingeben und bei kleiner Hitze
  stocken lassen, dabei die Pfanne ab und zu ein wenig bewegen. Das Omelette ist gut, wenn
  es sich vom Boden löst und die Oberfläche noch weich und cremig ist.

  Die Füllung in die Mitte des Omelettes legen und von beiden Seiten einklappen. Eventuell
  noch ein wenig mit Käse (z.B. Parmesan) gratinieren.
\end{recipe}
