\begin{recipe}{Pizza}{Von Lutz Moppert}
  \label{Pizza}
  \inglist[Für den Teig:]
  \ingredient{400 g Mehl}
  \ingredient{250 ml Wasser}
  \ingredient{21 g Hefe}
  \ingredient{\halb TL Salz}
  %\ingredient{1 Prise Zucker}
  \ingredient{Olivenöl}
  
  \inglist[Für die Sauce:]
  \ingredient{1 kleine Dose Tomaten}
  \ingredient{3 EL Knoblauchöl}
  \ingredient{1 Zweig Thymian}
  \ingredient{2 TL Oregano}
  \ingredient{1 TL Zucker}
  \ingredient{1 EL Tomatenmark}
  
  \steps 
  Das Knoblauchöl mit Salz, Zucker und den Dosentomaten aufkochen und ca. eine Stunde
  köcheln lassen. Am Ende Thymian, Chili und Oregano noch 5 Minuten mit ziehen lassen, den
  Thymianzweig wieder entfernen und mit Tomatenmark binden.

  Das Mehl mit dem Salz mischen, in eine Rührschüssel geben. Die Hefe im lauwarmen Wasser
  auflösen, zu dem Mehl geben und mit dem Handrührgerät mindestens 10 Minuten kneten.  Den
  Teig in vier Portionen Teilen (ca. 170 g pro Pizza), zu Kugeln formen und unter einem 
  Handtuch gehen lassen.
  
  Den Teig mindestens eine halbe Stunde gehen lassen, dann noch einmal mit der Hand
  durchkneten bis das Olivenöl homogen verteilt ist, anschließend ausrollen und wieder 10
  Minuten gehen lassen.

  Pizza mit Tomatensauce bestreichen, mit Käse bestreuen und nach Belieben mit fr.
  Champignons belegen. Die Pizza im vorgeheizten Backofen auf voller Stufe auf dem Boden
  liegend oder auf einem Pizzastein fertig backen. (z.B. 12 Min bei 275 \celsius
  Unterhitze ohne Umluft)

  \textbf{Zutaten zum mitbacken:} 
  Paprika, Oliven, Spinat, Broccoli, Zwiebeln, Peperoni oder Artischocken.

  \textbf{Zutaten für nachher:} 
  Ruccola, Schinken, Parmesan, Salami, Thunfisch,   Hähnchen- oder Putenfleisch, Mais oder Spargel.

\end{recipe}
