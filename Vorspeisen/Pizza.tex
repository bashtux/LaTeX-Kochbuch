\begin{recipe}{Pizza}{Von Lutz Moppert}
  \label{Pizza}
  \inglist[Für den Teig:]
  \ingredient{400 g Mehl}
  \ingredient{250 ml Wasser}
  \ingredient{10 g Hefe\index{Hefe}}
  \ingredient{\halb TL Salz}
  %\ingredient{1 Prise Zucker}
  \ingredient{Olivenöl}

  \inglist[Für die Sauce:]
  \ingredient{3 EL Olivenöl}
  \ingredient{1 EL Zucker}
  \ingredient{1 Zehe Knoblauch}
  \ingredient{1 Zweig Thymian}
  \ingredient{1 Pckg Pomito}
  \ingredient{2 TL Oregano}
  \ingredient{1 EL Tomatenmark}

  \steps

  Das Olivenöl mit dem Zucker, Salz und dem Thymian erhitzen bis der Zucker
  karamelisiert. Knoblauch und Thymian wieder entfernen und vorsichtig das
  Tomatenpürree dazu geben. Das Ganze eine Stunde köcheln lassen. Am Ende mit
  Tomatenmark binden.\\

  Die Hefe im Wasser auflösen und die Hälfte des Mehls unterrühren. Das
  restliche Mehl darüber geben (noch nicht verrühren) und mit dem Salz mischen.
  Nach ca.  15 Minuten mit dem Handmixer mindestenz 10 Min. durchkneten. Den
  Teig in Portionen Teilen (ca. 165 g pro Pizza), zu Kugeln formen und mit Olivenöl 
  einreiben.\\

  Den Teig mindestens eine halbe Stunde gehen lassen, dann noch einmal mit der
  Hand durchkneten bis das Olivenöl homogen verteilt ist, anschließend
  ausrollen und wieder 10 Minuten gehen lassen.\\

  Den Teig mit Tomatensauce bestreichen, mit Käse bestreuen und nach Belieben
  mit Zutaten belegen. Die Pizza im vorgeheizten Backofen auf voller Stufe
  auf dem Boden liegend oder auf einem Pizzastein fertig backen.\\

  Die angegeben Mengen sind für 4 Pizzen, folgende Tabelle zeigt die
  entsprechenden Umrechnungen für andere Mengen:

  \begin{tabular}[h]{l|l|l|l|l|l}
    \textbf{Anzahl} & 2 & 4 & 6 & 8 & 10 \\
    \hline
    \textbf{Mehl} & 200 g & 400 g & 600 g & 800 g & 1000 g \\
    \hline
    \textbf{Wasser} & 125 ml & 250 ml & 375 ml & 500 ml & 625 ml \\
  \end{tabular}

  \textbf{Zutaten zum mitbacken:}
  Paprika, Oliven, Spinat, Broccoli, Zwiebeln, Peperoni oder Artischocken.

  \textbf{Zutaten für nachher:}
  Ruccola, Schinken, Parmesan, Salami, Thunfisch,   Hähnchen- oder
  Putenfleisch, Mais oder Spargel.

\end{recipe}
