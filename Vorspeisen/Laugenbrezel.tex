\begin{recipe}{Laugenbrezel}{Eigenkreation}
  \label{Laugenbrezel}
  \inglist[Teig]
  \ingredient{1 kg Mehl}
  \ingredient{280 ml Wasser}
  \ingredient{250 ml Milch}
  \ingredient{150 g Butter}
  \ingredient{2 TL Salz}
  \ingredient{42 g Hefe\index{Hefe}}

  \inglist[Lauge]
  \ingredient{1\halb l Wasser}
  \ingredient{3 EL Natron}

  \steps

  Die Butter in 250 ml Wasser schmelzen lassen, anschließend mit kalter Milch
  mischen (die Flüssigkeit sollte handwarm sein) und die Hefe darin auflösen.
  Ein Prise Zucker und etwas Mehl zugeben und ca. zehn Minuten warten, bis die
  Hefe aktiv wird. In der Zwischenzeit das Salz im restlichen Wasser auflösen,
  zusammen mit dem restlichen Mehl auf den Vorteig geben und solange kneten bis
  ein geschmeidiger Teig entsteht.

  Den Teig zugedeckt ein Stunde gehen lassen, dann zu einer Rolle formen und in
  Stücke teilen. Die Teigteile in Stangen oder Brezeln formen. Das Wasser für
  die Lauge zum kochen bringen, und das Natron zugeben (vorsicht, das Schäumt)
  und die Teigteile nacheinander hineinlegen. Sobald das Teigstück nach einigen
  Sekunden oben schwimmt, wieder rausnehmen.

  Die vorgegarten und abgetropften Teigstücke auf ein Backblech legen und
  nach belieben mit grobem Salz oder Käse bestreuen. Im Backofen bei 160
  \celsius für 25 Minuten backen.

\end{recipe}
