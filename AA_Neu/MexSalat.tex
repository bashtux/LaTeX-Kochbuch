\documentclass[10pt, parskip, a4paper, headsepline, oneside]{scrartcl}

	\usepackage{ngerman}
	\usepackage[latin1]{inputenc}
	\usepackage{marvosym}
	\usepackage[colorlinks, linkcolor=blue]{hyperref}
	\usepackage{multicol}
	
	\usepackage{scrpage2}
	\ihead[]{\footnotesize Rezeptliste}
	\chead[]{\footnotesize }
	\ohead[]{\footnotesize }
	\cfoot[]{\footnotesize\pagemark}
	\pagestyle{scrheadings}
	
	\newcommand{\gradc}{$^\circ$C~}
	\DeclareRobustCommand*{\ing}[3]{#2 #3\\}
	\newenvironment{inglist}{
		\begin{addmargin}[1cm]{5cm}
			\begin{multicols}{2}
				\footnotesize
	}{
			\end{multicols}
		\end{addmargin}
		\normalsize
	}

\begin{document}

\section{Mexikanischer Salat}
	\begin{inglist}
		\ing{F}{500 g}{Gehacktes}
		\ing{G}{1}{Zwiebel}
		\ing{G}{4-6}{Tomaten}
		\ing{M}{2 Becher}{Schmand}
		\ing{K}{1 Flasche}{Salsa}
		\ing{K}{200 g}{K�se}
		\ing{K}{1 T�te}{Tacco Chips}
	\end{inglist}
	Gehacktes anbraten, die fein geschnittene Ziebeln kurz anziehen lassen. Das 
	ganze mit Chilli Gew�rzmischung und 1/2 Flasche Salsa w�rzen und kalt 
	stellen.
	
	Den Eisbergsalat klein schneiden, die kalte Fleischmischung, restliche Salsa,
	geschnittene Tomaten Schmand und den gersapelten K�se ebenfalls untermischen.
	Ca. 4-5 Stunden ziehen lassen und kurz vor dem Servieren mit zerschlagenen 
	Tacco-Chips garnieren.

\end{document}
