\begin{recipe}{Paprika Huhn}{Eigenkreation}
  \label{Paprikahuhn}
  \inglist
  \ingredient{1 Maishuhn/Biohuhn}
  \ingredient{3 Paprika}
  \ingredient{2 Zwiebeln}
  \ingredient{3 Möhren}
  \ingredient{1 Scheibe Sellerie}
  \ingredient{2 Knoblauch Zehen}
  \ingredient{1 kleine Dose Tomaten}
  \ingredient{1 Becher Schmand}
  \ingredient{1 EL Tomatenmark}
  \ingredient{1 TL Zucker}
  \ingredient{1 EL Olivenöl}
  \ingredient{1 Stck. Butter}
  \ingredient{1 Glas Rotwein}
  \ingredient{1/2 l Hühnerbrühe}
  \steps

  Das Huhn von allen Seiten anbraten und warm stellen. Zwiebeln, Möhren und
  Sellerie fein würfeln und ebenfalls kräftig anbraten. Das Tomatenmark, den
  gewürfelten Knoblauch und den Zucker unterrühren und das Stück Butter dazu
  geben. Mit Rotwein ablöschen und trocken kochen, dann die Brühe zugeben und
  die Hühnerteile darauf verteilen. Das ganze in den Ofen schieben und bei 150
  \celsius ca. 45 Min. schmoren.

  In der Zwischenzeit die Paprika in Streifen schneiden und anbraten. Am Ende
  der Garzeit die Hähnchenteile wieder warm stellen. Die Sauce durch ein Sieb
  passieren und auf die hälfte einkochen lassen. Die Paprika und den Schmand
  unterrühren. Die Sauce mit Salz, Pfeffer und Zucker abschmecken. Die
  Hähnchenteile wieder in die Sauce geben und warm werden lassen.

\end{recipe}
