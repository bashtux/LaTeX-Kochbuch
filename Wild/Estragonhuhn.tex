\begin{recipe}{Estragonhuhn}{nach Alfons Schuhbeck?}
  \label{Estragonhuhn}
  \inglist
  \ingredient{1 große Dose Tomaten}
  \ingredient{7 El Olivenöl}
  \ingredient{1 weiße Zwiebel}
  \ingredient{3 Schalotten}
  \ingredient{6 Zehen Koblauch}
  \ingredient{1 Huhn (ca. 1,5 kg)}
  \ingredient{20 g Butter}
  \ingredient{2 Lorbeerblätter}
  \ingredient{4 Stängel Estragon}
  \ingredient{25 ml Estragonessig}
  \ingredient{100 ml Weißwein}
  \ingredient{150 ml Geflügelfond}
  \ingredient{1 TL Dijonsenf}
  \ingredient{1 EL Tomatenmark}
  \ingredient{250 g Sahne}
  \ingredient{200 g kleine Steinpilze}
  \ingredient{1 TL Zitronensaft}

  \steps
  Tomaten kreuzweise einritzen, mit heißem Wasser überbrühen, häuten,
  halbieren, entkernen und grob zerkleinern. I EL Olivenöl erhitzen. Tomaten
  darin mit I Prise Zucker, Salz und Pfeffer langsam zu einem Mus einkochen
  lassen.

  Weiße Zwiebel schälen und würfeln. Schalotten schälen.  Knoblauchzehen
  ungeschält etwas andrücken. Das Huhn in 8 Stücke teilen. Mit Salz und Pfeffer
  würzen.

  Backofen auf 190 \celsius (Umluft 170 \celsius) vorheizen. Butter und 2 EL
  Olivenöl in einem Schmortopf erhitzen, Die Hühnerteile darin von allen Seiten
  anbraten.  Zwiebelwürfel, Knoblauch, Lorbeerblätter und 2 Stängel Estragon
  zugeben und ebenfalls anbraten. Danach im Backofen auf der mittleren Schiene
  etwa 20 Minuten schmoren. Nach 10 Minuten Estragonessig darüberträufeln und
  weiterschmoren.

  Geflügelteile aus dem Schmortopf heben und warm stellen. 2 EL Olivenöl zum
  Schmorfond geben. Schalotten hinzufügen und anbraten. Mit Weißwein ablöschen.
  Geflügelfond dazugießen. Senf und Tomatenmark unterrühren. Alles um die
  Hälfte einkochen lassen. Das eingekochte Tomatenmus und die Sahne
  unterrühren. Erneut würzen. Geflügelteile wieder hineingeben und im Ofen auf
  der mittleren Schiene weitere 15 Minuten schmoren.

  Pilze mit Zitronensaft beträufeln, würzen und im restlichen Olivenöl (2 EL)
  etwa 4 Minuten braten.  Pilze etwa 5 Minuten vor Ende der Garzeit zum Huhn
  geben und mitschmoren.  Estragonblättchen von den übrigen beiden Stängeln
  zupfen und fein schneiden.  Huhn aus dem Ofen nehmen, erneut abschmecken und
  mit Estragonblättchen bestreut servieren. Dazu passt Baguette.

\end{recipe}
