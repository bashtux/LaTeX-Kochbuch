\begin{recipe}{Hähnchenbrust auf Kürbispüree}{nach Johann Lafer}
  \inglist[Für das Püree]
  \ingredient{500 g Muskat-Kürbis}
  \ingredient{500 g Mehligkochende Kartoffeln}
  \ingredient{200 ml Milch}
  \ingredient{100 g Butter}
  \ingredient{150 ml Sahne}

  \inglist[Für das Hähnchen]
  \ingredient{2 Hühnerbrustfilets}
  \ingredient{2 Zweige Thymian}
  \ingredient{2 Knoblauchzehen}
  \ingredient{150 ml Geflügelfond}
  \ingredient{50 g Butter}

  \steps
  Die Kartoffeln in grobe Würfel schneiden undin ca. 15-20 Minuten weich 
  dämpfen. Den Kürbis bei Bedarf schälen, entkernen, in grobe Würfel
  schneiden und in der Milch ca. 20 Minuten weich kochen.

  Die Kartoffeln etwas ausdampfen lassen, durch die Presse drücken und die
  gewürfelte Butter untermengen. Die Sahne aufkochen und kräftig mit Muskat,
  Salz und Pfeffer würzen. Die Kürbismilch pürieren und zusammen mit der Sahne
  zu den Kartoffeln geben.

  Für die Hähnchen den Backofen auf 140 \celsius aufheizen. Das Fleisch salzen
  und Pfeffern und in einer Pfanne zusammen mit den geschälten, ganzen
  Knoblauchzehen und dem Thymian rundrum anbraten. Anschließen im Ofen in ca.
  10-15 Minuten gar ziehen.

  Die Gewürze und die Hähnchen aus der Pfanne nehmen und der Bratensatz mit der
  Brühe ablöschen. Die Butter oder etwas Mondamin zum Binden verwenden und die
  Sauce mit Salzund Pfeffer abschmecken.
\end{recipe}
