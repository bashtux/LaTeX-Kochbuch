\begin{recipe}{Wildschwein-Rücken}{}
  \label{Wildschwein-R"ucken}
  \inglist[Für die Marinade:]
  \ingredient{100 g Zwiebeln}
  \ingredient{50 g Schalotten}
  \ingredient{2 Knoblauchzehen}
  \ingredient{50 g Sellerie}
  \ingredient{50 g Petersilienwurzel}
  \ingredient{2 Thymianzweige}
  \ingredient{2 Lorbeerblätter}
  \ingredient{4 Körner schwarzer Pfeffer}
  \ingredient{4 Körner roter Pfeffer}
  \ingredient{2 Gewürznelken}
  \ingredient{2 l guter Rotwein}
  \ingredient{1 Wildschwein-Rücken\index{Wild>Wildschwein}}

  \inglist[Für den Braten:]
  \ingredient{Salz}
  \ingredient{frischer Pfeffer}
  \ingredient{2 Zwiebeln}
  \ingredient{1 Bund Suppengrün}
  \ingredient{6 Wacholderbeeren}
  \ingredient{\viertel l Sahne}
  \ingredient{\viertel l Wildfond}

  \steps
  Das Gemüse und die Gewürze mit Olivenöl anschwitzen und mit dem Rotwein ablöschen.
  Anschließend noch gut \halb Stunde köcheln und dann abkühlen lassen. Den
  Wildschwein-Rücken in die kalte Beize legen, er sollte vollständig bedeckt sein.
  Mindestens 12 Stunden durchziehen lassen.

  Das Fleisch aus der Beize nehmen und trocken tupfen. Mit Salz und Pfeffer einreiben. Im
  Bräter mit etwas Öl von allen seiten scharf anbraten. Das Gemüse klein schneiden und
  kurz mit braten.

  Den Backofen auf 150 \celsius vorheizen. Von der Beize ca. \halb l zum ablöschen in den
  Bräter geben. Den Wild Fond und die zerdrückten Wacholderbeeren dazu und das ganze für
  ca. 2 Stunden im Ofen schmoren lassen.

  Am Ende den Rücken aus dem Bräter nehmen und ca. 10 Minuten ruhen lassen. In der
  Zwischenzeit Sauce und Gemüse durch ein Sieb schütten. Mit Sahne, Pfeffer, Salz, Chili,
  Tomatenmark und Zucker abschmecken und mit Mondamin oder Butter abbinden.
\end{recipe}
