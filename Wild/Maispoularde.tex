\begin{recipe}{Geschmorte Poularde}{frei nach Dieter Müller}
  \label{Geschmorte Poulardenbrust}
  \inglist[Für die Poularde:]
  \ingredient{1 Maispoularde\index{Gefl"ugel>Poularde}}
  \ingredient{2 Thymianzweige}
  \ingredient{2 Rosmarinzweige}
  \ingredient{300 ml dunkler Geflügelfond}
  \ingredient{20 g kalte Butterwürfel}
  \ingredient{Portwein}

  \inglist[Für das Spinatflan:]
  \ingredient{350 g Blattspinat\index{Gem"use>Spinat}}
  \ingredient{2 Eier}
  \ingredient{2 EL Sahne}
  \ingredient{\viertel Knoblauchzehe}
  \ingredient{10 g Butter}

  \steps
  Das Fleisch von der Karkasse lösen (die Knochen können für eine spätere
  Verwendung eingefroren werden), mit Salz und Pfeffer würzen und auf der
  Hautseite auslassen. Zum bräunen kurz wenden und anschließend in einem dicht
  schießenden Schmortopf mit 4 EL Geflügelbrühe im vorgeheizten Ofen bei 190
  \celsius ca. 12 Minuten garen.

  Den Spinat in Salzwasser blanchieren, abschrecken und gut ausdrücken. In der
  gebräunten Butter anschwenken und mit Eiern, Sahne und Knoblauch fein
  pürieren und mit Salz und Pfeffer würzen. Die Spinatcreme in kleine
  Auflaufformen füllen und in ein mit warmen Wasser gefülltes Backblech
  stellen. Im Ofen bei 170 \celsius pochieren.

  Den Portwein stark einkochen und den restlichen Geflügelfond und den
  Schmorsud hinzugeben. Die Sauce einkochen, mit der kalten Butter leicht
  Binden und zum Fleisch und dem gestürzten Spinatflan servieren.
\end{recipe}
