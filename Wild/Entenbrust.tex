\begin{recipe}{Gebratene Entenbrust}{frei nach Alfons Schuhbeck}
  \label{Gebratene Entenbrust}
  \inglist
  \ingredient{1 Barbarieentenbrust\index{Gefl"ugel>Entenbrust}}
  \ingredient{Portwein}
  \ingredient{kalte Butter}
  \ingredient{Kaffeesalz}

  \steps
  Den Backofen auf 100 \celsius vorheizen. Die Entenbrüste von Sehnen befreien, eventuell
  vorhandene Federkiele mit einer Pinzette heraus zupfen, und die Hautseite rautenförmig
  bis zum Rand einritzen. Die Brust mit der Haut nach unten in eine kalte Pfanne legen,
  diese langsam erhitzen und bei mittlerer Hitze die Haut knusprig anbraten, dabei das
  Fleisch zwischendurch kurz wenden, damit die Haut wieder abkühlen kann und auch die
  andere Seite leicht anbrät.

  Nach etwa 5 Minuten ein Gitter in die Pfanne stellen, die Entenbrust darauf legen und
  das Ganze für ca. 1 Stunde in den Ofen stellen. Anschließend das Fleisch salzen (z.B mit
  Kaffeesalz oder Fleur de sel), pfeffern und zur Seite stellen. Den Bratsud mit Portwein
  auffüllen, einkochen lassen und mit der kalten Butter leicht binden. Nach belieben noch
  etwas Rotwein dazu geben und mit Rübenkraut abschmecken, dann mit Kaffeesalz und Pfeffer
  würzen und zur Entenbrust servieren.
\end{recipe}
