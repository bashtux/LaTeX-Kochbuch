\begin{recipe}{Tomatensuppe}{nach Dieter Müller}\label{Tomatensuppe}
  \inglist
  \ingredient{1 kg Tomaten}
  \ingredient{1 Karotte}
  \ingredient{2 Zwiebeln}
  \ingredient{1 Knoblauchzehe}
  \ingredient{1 kleines Stück Sellerie}
  \ingredient{40 g Butter}
  \ingredient{800 ml Brühe}
  \ingredient{2 Eiweiß}
  \ingredient{10 Eiswürfel}
  \ingredient{3 EL Tomatenmark}
  \ingredient{2 cl Gin}

  \steps

  Karotte, Sellerie, Zwiebeln und den Knoblauch schälen und in gleich große
  Würfel schneiden. Die Butter in einer Pfanne auslassen und das Gemüse darin
  andünsten. Mit den Tomaten und dem Geflügelfond auffüllen. Bei milder Hitze
  15 - 20 Minuten köcheln lassen. Vom Herd nehmen und mindestens 3 Stunden sehr
  gut durchkühlen.

  Eiweiß mit den Eiswürfeln und dem Tomatenmark vermischen und unter die kalte
  Suppe rühren. Auf dem Herd bei milder Hitze unter ständigem, vorsichtigen
  Rühren langsam zum Sieden bringen und 15 Minuten ziehen lassen. Anschließend
  durch ein feines Sieb passieren. Mit Salz, Pfeffer, Zucker und Gin
  abschmecken und nochmals kurz erhitzen.

\end{recipe}
