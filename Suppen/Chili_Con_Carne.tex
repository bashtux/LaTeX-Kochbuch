\begin{recipe}{Chili Con Carne}{nach Lutz Moppert}
  \inglist
  \ingredient{500 g Gehacktes\index{Fleisch>Gehacktes}\index{Hackfleisch}}
  \ingredient{1 große Zwiebel}
  \ingredient{1 EL Olivenöl}
  \ingredient{1 EL Tomatenmark}
  \ingredient{100 ml Rotwein}
  \ingredient{2 Zehen Knoblauch}
  \ingredient{2 Paprika\index{Gem"use>Paprika}}
  \ingredient{1 gr. Dose Tomaten\index{Gem"use>Tomaten}}
  \ingredient{1 kl. Dosen Kidney Bohnen\index{Gem"use>Bohnen}\index{Bohnen>Kidney}}
  \ingredient{2 getrocknete Chili}
  \ingredient{2 Lorbeerblätter}
  \ingredient{\halb TL Zucker}
  \ingredient{\halb TL Kreuzkümmel}
  \ingredient{1 TL Oregano}
  
  \steps
  Das Gehackte anbraten bis alle Flüßigkeit verdampft und das Fleisch leicht gebräunt ist.
  Dann die Zwiebel dazu geben und kurz andünsten. 
  
  Das Tomatenmark und ein wenig Olivenöl unterrühren und kurz anrösten, dann mit Rotwein
  ablöschen und wieder warten, bis die gesamte Flüßigkeit verdunstet ist. Erst jetzt die
  Tomaten zufügen und Alles eine gute Stunde köcheln lassen.
  
  Die Paprika schälen, klein schneiden, die Kidney Bohnen abtropfe und beides zu dem Sugo
  geben. Den Knoblauch in feine Scheiben schneiden und zusammen mit den restlichen
  Gewürzen ebenfalls beifügen. 
  
  Nach ca. 10 Minuten das Lorbeerblatt wieder entfernen, mit Salz und Pfeffer abschmecken
  und zur Milderung der Schärfe nach belieben mit Joghurt und Baguette servieren.
\end{recipe}
