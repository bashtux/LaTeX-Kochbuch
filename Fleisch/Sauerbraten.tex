\begin{recipe}{Rheinischer Sauerbraten}{frei nach Johann Lafer}
  \label{Sauerbraten}
  \inglist[Für die Marinade:]
  \ingredient{1,5 kg Pferdebraten\index{Fleisch>Pferd}}
  \ingredient{1 Möhre}
  \ingredient{1 Stange Lauch}
  \ingredient{\viertel Knollensellerie}
  \ingredient{2 Zwiebeln}
  \ingredient{750 ml Rotwein}
  \ingredient{250 ml Rotweinessig}
  \ingredient{1 EL Wachholderbeeren}
  \ingredient{3 Nelken}
  \ingredient{1 EL schwarze Pfefferkörner}
  \ingredient{3 Lorbeerblätter}
  \ingredient{100 g Zucker}
  
  \inglist[Für die Sauce:]
  \ingredient{\halb TL Korianderkörner}
  \ingredient{1 TL Pimentkörner}
  \ingredient{1 Splitter Zimtrinde}
  \ingredient{l2 Backpflaumen}
  \ingredient{6 Gewürz-Lebkuchen}
  \ingredient{1 TL Zartbitterkuvertüre}
  
  \steps
  Das Wurzelgemüse waschen, putzen und in grobe Würfel schneiden. Den Wein, den Essig, die
  gewürze und das Gemüse in einem Topf kurz aufkochen und wieder kalt werden lassen. Dann
  das Fleisch darin einige Tage marinieren.
  
  Das Fleisch aus der Marinade nehmen, trocken tupfen und in einem Bräter anbraten. Mit
  Salz und Pfeffer würzen und aus dem Bräter nehmen. Das Gemüse aus der  Marinade nehmen
  und im Bräter anrösten, Tomatenmark zufügen und leicht anbraten. 
  
  Den restlichen Sud hinzufügen, die Saucengewürze und die zerbröselten Lebkkuchen und das
  Fleisch hineingeben und zugedeckt 3 - 4 Stunden im Backofen bei 180 \celsius schmoren.
  Dann den Sud abseiehn, einkochen und als Grundlage für die Sauce verwenden. Diese mit
  Salz, Pfeffer und Rübenkraut abschmecken.
\end{recipe}
