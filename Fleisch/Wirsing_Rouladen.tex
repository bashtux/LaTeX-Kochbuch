\begin{recipe}{Wirsing-Rouladen}{Utas Leibspeise}
  \label{Wirsing-Rouladen}
  \inglist
  \ingredient{1 Wirsing}
  \ingredient{75 g Speck}
  \ingredient{\viertel l Brühe}
  \ingredient{1 EL Olivenöl}
  \ingredient{1 EL Tomatenmark}

  \inglist[Für die Füllung]
  \ingredient{1 Zwiebel}
  \ingredient{500 g Hackfleisch\index{Fleisch>Gehacktes}\index{Hackfleisch}}
  \ingredient{2 Eier}
  \ingredient{1 Bund Petersilie}
  \ingredient{3 EL Semmelbrösel}
  \ingredient{1 Prise Muskatnuss}

  \steps
  Vom Wirsing die Blätter abtrennen und in reichlich Salzwasser blanchieren. Die dicken
  Strunke halbieren und die Blätter nach Größe sortiert zusammenlegen. Petersilie fein 
  hacken und mit den Eiern und den Semmelbrösel verkneten. Die Zwiebel in Würfel
  schneiden, kurz andünsten und zu der Hackfleisch-Masse geben, das ganze mit Salz, Pfeffer 
  und Muskat kräftig würzen.

  Die Masse auf die Kohlblätter legen und zusammenrollen, evtl. mit Garn oder ähnlichem 
  fixieren.

  Den Speck in Würfel schneiden, Öl in einem großen Topf erhitzen und den Speck darin 
  auslassen. Die Rouladen in dem Fett ringsum braun anbraten und mit der Fleischbrühe 
  ablöschen. Das Tomatenmark einrühren und das Ganze ca. 30 Minuten schmoren lassen.
  
\end{recipe}
