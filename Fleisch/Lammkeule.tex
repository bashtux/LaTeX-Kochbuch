\begin{recipe}{Lammkeule}{nach Dieter Müller}
  \label{Lammkeule}
  \inglist[Lammhaxe]
  \ingredient{1 Lammkeule\index{Fleisch>Lamm}}
  \ingredient{300 g Mirepoix}
  \ingredient{1 EL Tomatenmark}
  \ingredient{300 ml Rotwein}
  \ingredient{\halb l Fond}
  \ingredient{2 Knoblauchzehen}
  \ingredient{2 Zweige Bohnenkraut}
  \ingredient{1 Zweig Thymian}
  \ingredient{1 Zweig Rosmarin}
  \ingredient{10 Pfefferkörner}
  \ingredient{10 Korianderkörner}
  \ingredient{2 Pimentkörner}
  \ingredient{2 Lorbeerblätter}
  \ingredient{1 Sternanis}
  \ingredient{1 TL gehacktes Bohnenkraut}
  
  \inglist[Paprikasauce]
  \ingredient{1 rote Paprika}
  \ingredient{100 ml Olivenöl}
  \ingredient{Tabasco}
  
  \steps
  Die Lammkeule im Bräter anbraten, herausnemen und das Mirepoix anbraten. Tomatenmark zum
  Gemüse geben, kurz anrösten und dann nach und nach mit Rotwein ablöschen. Das Fleisch,
  den Fond und die Kräuter in den Bräter geben und das Ganze im Ofen bei 200 \celsius ca.
  zwei Stunden schmoren lassen. 
  
  Anschließend die Keule warm stellen und den Sud durch ein
  Sieb passieren und etwas einkochen. Mit gehacktem Bohnenkraut, Salz und Pfeffer
  abschmecken.

  Für die Sauce die Paprika im Ofen schwarz werden lassen, bis sich die Haut gut abziehen
  lässt. Das Paprikafleisch mit dem Öl fein pürieren und mit Pfeffer, Salz und Tabasco
  abschmecken.
\end{recipe}
