\begin{recipe}{Gulasch}{inspiriert von Alfons Schuhbeck}
  \label{Gulasch}
  \inglist
  \ingredient{1 kg Schweineschulter\index{Fleisch>Schweineschulter}\index{Schweinefleisch}}
  \ingredient{3 Zwiebeln}
  \ingredient{2 EL Tomatenmark}
  \ingredient{500 ml Geflügelfond\index{Gefl"ugelfond}}
  \ingredient{1 EL Paprikapulver}
  \ingredient{2 Knoblauchzehen}
  \ingredient{1 TL Majoran}
  \ingredient{1 Lorbeerblatt}
  \ingredient{1 TL Chilipulver}
  \ingredient{200 g Schmand}

  \steps
  Das Fleisch in 3 cm große Würfel schneiden. Die Zwiebeln schälen, halbieren und quer in
  dünne Scheiben schneiden.

  1 - 2 EL Öl in einem Topf erhitzen, das Fleisch darin bei milder Hitze portionsweise
  rundum anbraten und wieder herausnehmen. Die Zwiebelscheiben im verbliebenen Fett bei
  milder Hitze leicht andünsten. Das Tomatenmark hinzugeben, kurz anrösten und mit der
  Hühnerbrühe (siehe \pageref{Geflügelfond}) ablöschen. Das Fleisch wieder hinzufügen und
  bei milder Hitze mit aufgelegtem Deckel 2 Stunden schmoren. Nach 1 Stunde den Deckel
  abnehmen, damit die Sauce etwas einkocht.

  Das Paprikapulver mit etwas Wasser anrühren, den Knoblauch schälen und mit den anderen
  Gewürzen fein Hacken. Ungefähr 5 bis 10 Minuten vor Ende der Garzeit die Gewürze in das
  Gulasch rühren, fertig garen und mit Salz, Chili und Pfeffer abschmecken.

  Das Gulasch mit dem Schmand, der noch mit Zitronensaft und etwas Paprika- oder
  Chilipuver verrührt werden kann, servieren.
\end{recipe}
