\begin{recipe}{Rinderschmorbraten}{nach Christian Rach}
  \inglist[Für die Marinade:]
  \ingredient{1 Karotte}
  \ingredient{2 Zwiebeln}
  \ingredient{100 g Knollensellerie}
  \ingredient{1 Fenchelknolle}
  \ingredient{3 Knoblauchzehen}
  \ingredient{3 Lorbeerblätter}
  \ingredient{1 Nelke}
  \ingredient{1 TL Pfefferkörner}
  \ingredient{3 EL Balsamico-Essig}
  \ingredient{1 l Rotwein}
  \ingredient{4 EL Olivenöl}

  \inglist[Für den Braten:]
  \ingredient{1,3 kg Rinderbraten\index{Rind>Braten}\index{Fleisch>Rinderbraten}}
  \ingredient{4 EL Olivenöl}
  \ingredient{2 EL Tomatenmark}
  \ingredient{1 EL Mehl}
  \ingredient{500 ml Rinderfond}

  \steps
  Die Karotte, Zwiebeln und Knollensellerie schälen und ebenso wie den Fenchel waschen.
  Das ganze Gemüse in feine Scheiben bzw. Würfel schneiden. Knoblauchlzehen schälen und
  leicht andrücken. Gemüse, Gewürze, Balsamico, Wein und Öl in einer passenden Schüssel
  mischen und den Rinderbraten darin so einlegen, dass er vollständig bedeckt ist. Die
  Schüssel mit Klarsichtfolie bedecken und kalt stellen. Den Braten 4-5 Tage marinieren
  und dabei zweimal wenden.

  Das Fleisch aus der Marinade heben und gut trocken tupfen. Den Braten mit Salz und
  Pfeffer würzen und in 2 EL Olivenöl in einem passenden Bräter ringsherum anbraten.
  Fleisch wieder aus dem Bräter heben und das Gemüse im restlichen Öl anbraten.
  Röstaromen sind gewollt -- Es gibt kein Schwarz, höchstens dunkel Braun -- also
  keine Angst!

  Tomatenmark einrühren, Mehl darüber stäuben und kurz mit rösten bis das Tomatenmark
  leicht braun wird. Die Marinade durch eine Sieb dazu gießen, das Gemüse beiseite
  stellen. Die Flüßigkeit auf die hälfte einkochen, dann den Braten zurück in den Topf
  und verschließen. Das Ganze in dem auf 180 \celsius vorgeheizten Backofen etwa
  2\halb Stunden schmoren.

  Am Ende der Garzeit den Braten aus der Flüssigkeit heben und warm stellen. Den
  Schmorsud durch ein Sieb passieren und nochmals stark einkochen. Eventuell noch mit
  ein wenig kalter Butter oder Mehl-Butter binden und mit Salz, Pfeffer, Zucker und
  Balsamicoessig abschmecken. Den Braten in Scheiben schneiden und mit der Sauce
  servieren.  
\end{recipe}
