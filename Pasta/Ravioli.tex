\begin{recipe}{Ravioli}{frei nach Christian Rach}
  \label{Ravoli}
  \inglist[Für den Nudelteig]
  \ingredient{250 g Mehl}
  \ingredient{5 Eier}
  \ingredient{1 EL Olivenöl}

  \inglist[Für die Füllung]
  \ingredient{400 g Ruccola}
  \ingredient{250 g Schafkäse}
  \ingredient{2 Schalotten}
  \ingredient{1 Knoblauchzehe}
  \ingredient{50 g Butter}
  
  \inglist[Für die Sauce]
  \ingredient{1 Tomate}
  \ingredient{1 Koblauchzehe}
  \ingredient{1 Rosmarinzweig}
  \ingredient{6 EL Olivenöl}
  \ingredient{1 getrocknete Chili}
  \ingredient{Parmesan}
  \index{Pasta>Ravioli}
  
  \steps
  Ein Ei trennen und das Eiweiß zur Seite stellen. Das Mehl in eine Schüssel sieben und 
  eine Mulde in die Mitte drücken. Eier mit Öl verquirlen, in die Mulde geben und nach und 
  nach mit dem Mehl vermengen bis ein geschmeidiger Nudelteig entsteht (evtl. noch etwas 
  Wasser hinzugeben). Den Teig mit Öl einreiben und mindestens eine Stunde ziehen lassen.

  Die fein gewürfelten Schalotten in der Butter anschwitzen, den Knoblauch ebenfalls
  würfeln und ganz kurz mitbraten. Den Rucola fein schneiden und mit den Zwiebeln und dem
  zerbräselten Schafskäse mischen.

  Den Nudelteig zu langen Streifen ausrollen und eine Hälfte hiervon auf eine bemehlte
  Arbeitsfläche legen. Die Füllung Portionsweise (jeweils 1 EL) daraufgeben und die Ränder
  mit dem Eiweiß bestreichen. Die restlichen Streifen  über die anderen legen und die
  Ränder zu Ravioli zusammendrücken.

  Die Tomate häuten und filetieren, den Knoblauch in Scheiben schneiden und beides in
  Olivenöl zusammen mit dem Rosmarin  und der Chili anschwitzen. Parallel die Nudel in 
  reichlich Salzwasser 2-3 Minuten garen und anschließend im Öl schwenken. Mit Parmesan 
  servieren.
\end{recipe}


