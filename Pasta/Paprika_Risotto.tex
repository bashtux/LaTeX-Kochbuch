\begin{recipe}{Paprika-Risotto}{nach Lutz Moppert}
  \label{Paprika-Risotto}
  \inglist
  \ingredient{300 g Reis (Arborio)\index{Reisgerichte}}
  \ingredient{2 Schalotten}
  \ingredient{1 EL Öl}
  \ingredient{2 rote Paprika\index{Gem"use>Paprika}}
  \ingredient{\halb l Geflügelfond}
  \ingredient{200 ml Weißwein}
  \ingredient{1 EL Butter}
  \ingredient{100 g Parmesan}
  \index{Reis}\index{Risotto}

  \steps
  Die Schalotten schälen und fein würfeln. Den Reis in etwas Öl kurz anbraten, die
  Schalotten zugeben und glasig werden lassen. Mit dem Weißwein ablöschen und die
  Flüssigkeit vollständig verdunsten lassen. Den heißen Fond nach und nach zugeben und
  unter Rühren das Risotto gar ziehen.

  Die Paprika grob schälen und in einer Pfanne andünsten. Ungefähr die Hälfte wird in
  Würfel geschnitten und zur Seite gestellt, die restlichen Paprikastücke pürieren (falls
  an manchen Stücken noch etwas Schale dran ist, diese zum pürieren verwenden.

  Am Ende der Garzeit (ca. 30 Min.) das Paprika-Püree, die Paprika-Stücke, die Butter und
  den geriebenen Parmesan unter das Risotto mischen. Mit Salz und Pfeffer abschmecken und
  servieren.

\end{recipe}
