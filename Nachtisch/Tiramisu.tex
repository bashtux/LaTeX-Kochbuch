\begin{recipe}{Tiramisu}{nach Johann Lafer}
  \label{Tiramisu}
  \inglist[Grundzutaten:]
  \ingredient{3 frische Eier}
  \ingredient{80 g Zucker}
  \ingredient{20 g Vanillezucker}
  \ingredient{500 g Mascarpone}
  \ingredient{16 - 20 Löffelbiskuits}
  \ingredient{Kakaopulver}

  \inglist[Klassisches-Tiramisu:]
  \ingredient{3 cl Amaretto}
  \ingredient{150 ml Espresso}

  \inglist[Schokoladen-Tiramisu:]
  \ingredient{100 g Zartbitterschokolade}
  \ingredient{1 TL Zimtpulver}
  \ingredient{150 ml Milch}
  \ingredient{3 cl Kakaolikör}
  
  \steps 
  
  Die Eier trennen und das Eigelb mit der Hälfte des Zuckers auf dem Wasserbad schaumig
  schlagen.  Anschließend den Mascarpone unter heben. 

  \emph{Schokoladen-Tiramisu:} Die Schokolade auf dem Wasserbad schmelzen und zusammen mit
  dem Zimtpulver unter die Masse rühren.
  
  Das Eiklar mit einer Prise Vanillesalz steif schlagen und den restlichen Zucker nach und
  nach einrieseln lassen.  Den Eischnee vorsichtig unter die Mascarponemasse heben.

  \emph{Klassische Variante:} Den Espresso mit dem Amaretto mischen.
  
  \emph{Schokoladen-Tiramisu:} Die Milch erwärmen und 2 EL Kakaopulver und den Likör
  einrühren.

  Die Löffelbiskuit in eine Auflaufform legen, mit Flüssigkeit beträufeln und die
  Mascarpone Creme darüber verteilen.  Das ganze noch einmal mit einer zweiten Schicht
  wiederholen. Anschließend das Tiramisu 2 Stunden kalt stellen und vor dem Servieren mit
  Kakaopulver bestreuen.
\end{recipe}
