\begin{recipe}{Rotkohl}{frei nach Alfons Schuhbeck}
  \inglist
  \ingredient{800 g Rotkohl\index{Rotkohl}}
  \ingredient{1 EL Puderzucker}
  \ingredient{100 ml roter Portwein}
  \ingredient{200 ml Rotwein}
  \ingredient{125 ml Gemüsebrühe}
  \ingredient{1 Lorbeerblatt}
  \ingredient{5 Pimentkörner}
  \ingredient{\halb TL schwarzer Pfeffer}
  \ingredient{Zimtrinde}
  \ingredient{1 Schuss Grand Marnier}
  \ingredient{1 Scheibe Ingwer}
  \ingredient{3 Äpfel}
  \ingredient{2 EL kalte Butter}
  \ingredient{1 EL Balsamicoessig}
  \index{Gem"use>Rotkohl}

  \steps

  Den Rotkohl in feine Streifen hobeln. In einem Topf den Puderzucker hell
  karamellisieren lassen, Portwein und Rotwein dazu gießen und auf ein Drittel
  reduzieren lassen. Brühe und Rotkohl hinzufügen. Etwa 1\halb Stunden bei
  milder Hitze zugedeckt mehr ziehen als köcheln lassen und dabei des öfteren
  umrühren.

  Nach 1 Stunde das Lorbeerblatt einlegen, Pimentkörner, Pfeffer und Zimt in
  einen Einwegteebeutel füllen, verschließen und in das Kraut legen.

  Am Ende der Garzeit die Äpfel reiben und hinein rühren, Orangenschale und
  Ingwer einlegen. Das ganze einige Minuten darin ziehen lassen, danach mit dem
  Lorbeerblatt und dem Gewürzsäckchen entfernen und den Sud in eine Pfanne
  gießen.

  Den Sud stark einkochen und nach belieben mit etwas Orangenlikör abschmecken,
  dann die Butter hinein rühren. Mit Salz, Zucker und mildem Balsamico
  abschmecken und wieder zum Rotkohl geben.
\end{recipe}
