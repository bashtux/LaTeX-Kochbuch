\begin{recipe}{Knödel mit Wirsing}{nach Kolja Kleeberg}
  \inglist[Für die Knödel:]
  \ingredient{500 g Roggenbrot (75 \%)}
  \ingredient{40 g Speckwürfel}
  \ingredient{40 g Schalottenwürfel}
  \ingredient{400 ml lauwarme Milch}
  \ingredient{4 Eier}
  \ingredient{1 Prise	gemahlener Koriander, geriebene Muskatnuß, Pfeffer aus der Mühle}
  \ingredient{\halb Bund Petersilie}
  \ingredient{30 g gehackte Walnüsse}
  \ingredient{150 g Käse}

  \inglist[Für den Wirsing:]
  \ingredient{1 Kopf Wirsing}
  \ingredient{50 g Speckwürfel}
  \ingredient{200 ml Sahne}
  \ingredient{50 ml Gemüsefond}

  \steps
  Brot in 1 cm dicke Würfel schneiden, Speck und Schalotten glasig dünsten und die
  Walnüsse kurz mit rösten. Alles zum Brot geben und die lauwarme Milch, geschnittene
  Petersilie und Eier darüber gießen. Mit einem Kochlöffel vorsichtig durch heben, mit
  Salz, Koriander, Muskat und Pfeffer aus der Mühle würzen.  Die Mischung \halb bis 1
  Stunde quellen lassen. Den Käse in 1 cm große Würfel schneiden als Einlage (Kern) für
  die Knödel.

  Die Speckwürfel im heißen Wok ausbraten, dann den gezupften Wirsing kurz mitbraten und
  mit dem Geflügelfond ablöschen.  Sahne hinzugeben und 4 Minuten köcheln lassen. Mit Salz
  und Pfeffer aus der Mühle abschmecken.
\end{recipe}
