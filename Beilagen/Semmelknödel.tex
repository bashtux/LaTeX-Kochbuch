\begin{recipe}{Semmelknödel}{nach Johann Lafer}
  \inglist
  \ingredient{250 g alte Brötchen}
  \ingredient{200 ml Milch}
  \ingredient{25 g Butter}
  \ingredient{1 Zwiebel}
  \ingredient{1 Knolauchzehe}
  \ingredient{2 Eier}
  \index{Kn"odel}

  \steps
  Die Brötchen in kleine Würfel schneiden und in eine große Schüssel geben. Die Zwiebel
  und den Knoblauch schälen und fein würfeln. Die Butter in einer Pfanne auslassen und
  die Gemüsewürfel darin leicht anschwitzen.

  Die Milch in die Pfanne geben und ein paar Minuten köcheln lassen. Die Flüssigkeit mit
  Salz, Pfeffer und Muskatnuss kräftig abschmechen und das Ganze über die
  Brötchen schütten, die Eier hinzufügen und gut durchkneten.

  Ein Geschirrhandtuch nass machen, gut auswringen und die Knödelmasse darin zu einer
  Wurst formen. Die Enden mit Clips verschließen und den Kloß in reichlich Wasser ca. 10
  Minuten gar ziehen. Am Ende in Scheiben schneiden und mit reichlich Sauce servieren.

  %Bei Bedarf können die scheiben auch eingefroren oder einige Zeit im Kühlschrank
  %aufbewahrt werden. Diese dann einfach in der Pfanne mit ein wenig Butter kurz anbraten.
\end{recipe}
