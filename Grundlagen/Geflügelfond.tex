\begin{recipe}{Gefl"ugelfond}{Grundlage}
  \label{Geflügelfond}
  \inglist
  \ingredient{1,5 kg Suppenhuhn\index{Suppenhuhn}}
  \ingredient{200 ml Weißwein\index{Wei"swein}}
  \ingredient{1 Zwiebel}
  \ingredient{1 Karotte}
  \ingredient{1 Sück Knollensellerie}
  \ingredient{\halb Lauchstange}
  \ingredient{1 Knoblauchzehe}
  \ingredient{1 Lorbeerblatt}
  \ingredient{1 Thymianzweig}
  \ingredient{1 Nelke}
  \ingredient{2 Petersilienzweige}
  \ingredient{10 Pfefferkörner}

  \steps
  Das Suppenhuhn gründlich waschen, in einem großen Topf mit kaltem Wassser bedecken und 
  langsam zum kochen bringen. Das Huhn 1 Stunde leise sieden lassen.
  
  Zwiebeln, Karotten, Knoblauch und Sellerie schälen und grob zerkleinern. Zusammen mit den 
  Gewürzen in den Topf geben und das ganlze eine weitere Stunde sieden lassen.

  Den Fond durch ein Sieb passieren und erkalten lassen. Das Fleisch von Huhn kann für ein 
  anderes Gericht verwendet werden, am besten als Frikasssee oder mit Majonaise, das es 
  durch das kochen ein wenig trocken geworden ist.
  
  Um einen \textbf{dunklen Geflügelfond} herzustellen, die Knochen vom Huhn zuvor eine
  Stunde im Ofen bei ca. 150 \celsius rösten. Das zerkleinerte Gemüse im Topf anrösten,
  Tomatenmark zu geben und kurz mitrösten und erst dann zur Brühe zugeben.
\end{recipe}
